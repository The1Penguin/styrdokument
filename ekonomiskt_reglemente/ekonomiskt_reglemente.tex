%LaTeX-inställningar%%%%%%%%%%%%%%%%%%%%%%%%%%%%%%%%%%%%%%%%%%%%%%%%%%%
% Kompliera med pdflatex!!!
\documentclass[a4paper, 10pt]{article}

\usepackage{hyperref}
\usepackage[utf8]{inputenc}
\usepackage[T1]{fontenc}
\usepackage[swedish]{babel}
\usepackage{graphicx}
\usepackage{etoolbox}
%\usepackage{ae} %För riktiga fonts?
\usepackage{fancyhdr}
\usepackage{lastpage}
\topmargin -20.0pt
\headheight 56.0pt
\setcounter{secnumdepth}{5}
\input xintexpr.sty\relax
\def\roundandprint #1{\xinttheiexpr #1\relax }

%FYLL I VID ÄNDRINGAR%%%%%%%%%%%%%%%%%%%%%%%%%%%%%%%%%%%%%%%%%%%%%%%%%%
\newcommand{\updated}{2024-11-22} %Insert

%%%%%%%%%%%%%%%%%%%%%%%%%%%%%%%%
% PRISBASBELOPP
% sätt till nuvarande prisbasbelopp för att annotera med prisbasbelopp i spalten till höger.
% OBS detta ska vara avstängt för det faktiska reglementesdokumentet och när ändringar ska godkännas på sektionsmöten
%\newcommand{\nuvarandeprisbasbelopp}{47600}


\newcommand{\prisbasbelopp}[1]{
    #1 prisbasbelopp
    \ifdef{\nuvarandeprisbasbelopp}{
        \marginpar{
            \small{ \textbf{\roundandprint{\xintiexpr#1 * \nuvarandeprisbasbelopp\relax}kr}}
        }
    }{}
}
%Dokumentstart%%%%%%%%%%%%%%%%%%%%%%%%%%%%%%%%%%%%%%%%%%%%%%%%%%%%%%%%%
\begin{document}
\pagestyle{fancy}

%Header%%%%%%%%%%%%%%%%%%%%%%%%%%%%%%%%%%%%%%%%%%%%%%%%%%%%%%%%%%%%%%%%
\renewcommand{\headrule}{\vbox to 0pt{\hfill\hbox to 292pt{\hrulefill}}}
\lhead{
\raisebox{-25pt}[0pt][10pt]{\includegraphics[width=60pt]{../Datalogo.pdf}}
\parbox[b]{200pt}{
\textbf{Datateknologsektionen}\\
Chalmers studentkår\\
Ekonomiskt reglemente}}
\rhead{ \flushright
Sidan \thepage\ av \pageref{LastPage}\\
Uppdaterad \updated}


%Footer%%%%%%%%%%%%%%%%%%%%%%%%%%%%%%%%%%%%%%%%%%%%%%%%%%%%%%%%%%%%%%%%
\renewcommand{\footrulewidth}{\headrulewidth}
\lfoot{\flushleft Datateknologsektionen\\
    Rännvägen 8\\
    412 58 Göteborg}
\cfoot{}
\rfoot{ \flushright styret@dtek.se\\
www.dtek.se}
\newpage

%Titlepage%%%%%%%%%%%%%%%%%%%%%%%%%%%%%%%%%%%%%%%%%%%%%%%%%%%%%%%%%%%%%
\vspace*{\fill}
\begin{center}
{\Huge \textbf{Ekonomiskt reglemente för Datateknologsektionen}}\\
\includegraphics[width=300pt]{../Datalogo.pdf}
\\{\LARGE Chalmers, Göteborg}
\end{center}
\vspace*{\fill}
\begin{center}
{\LARGE Uppdaterad: \updated}
\end{center}
\vspace*{\fill}


\newpage
\setcounter{tocdepth}{1}
\tableofcontents
\newpage

%%%%%%%%%%%%%%%%%%%%%%%%%%%%%%
%Allmänt
%%%%%%%%%%%%%%%%%%%%%%%%%%%%%%
\section{Allmänt}
\subsection{Definitioner}
Kommittéer används som ett samlingsnamn för kommittéer, Talhenspresidiet, 
presidiet och sektionsstyrelsen.

\subsection{}
Detta reglemente har skapats för att skapa tydlighet över vad sektionens pengar får och inte får användas till. Det är därmed tänkt att vara ett hjälpmedel för styrelsen, revisorerna och sektionens kommittéer i deras arbete. Grundfilosofin är att sektionens pengar är till för sektionens samtliga medlemmar, vilket i praktiken innebär att arrangemang som inte är öppna för dessa inte börbelasta sektionens ekonomi. Viktigt att tänka på vid tolkandet utav reglementet är att det inte har utformats för att i onödan begränsa kommittéers möjligheter eller lust till roliga arrangemang och upptåg.
\subsection{Prisbasbelopp}
\begin{itemize}
  \item Det prisbasbelopp som skall användas under hela verksamhetsåret är det prisbasbelopp som är aktuellt vid verksamhetsårets början.
  \item Delar av prisbasbelopp skall avrundas till närmaste hundratal.
\end{itemize}
%%%%%%%%%%%%%%%%%%%%%%%%%%%%%%%%%%%%%
% Styrelsen
%%%%%%%%%%%%%%%%%%%%%%%%%%%%%%%%%%%%%
\section{Styrelsen}
\subsection{}
Ordförande är skyldig att
\begin{itemize}
  \item Teckna sektionens firma
  \item Till varje sektionsmöte kunna redogöra om sektionens verksamhet
  \item Se till att ordförande i varje kommitté har tillgång till och kunskap om sektionens stadgar, reglemente och ekonomiska reglemente
  \item Se till att ordförande i relevanta kommittéer skriver en verksamhetsrapport inför varje brytpunkt.
\end{itemize}
\subsection{}
Kassören är skyldig att
\begin{itemize}
  \item Teckna sektionens firma
  \item Se till att kassör i varje kommitté har tillgång till och kunskap om sektionens stadgar, reglemente och ekonomiska reglemente
  \item Fortlöpande kontrollera sektionens räkenskaper och bokföring
  \item I samråd med styrelsen upprätta budgetförslag till första ordinarie höstmötet
  \item Till varje sektionsmöte kunna redogöra för sektionens ekonomiska ställning
  \item Utbilda nya sektionsfunktionärer i hur sektionens bokförings och redovisningssystem skall användas
\end{itemize}

%%%%%%%%%%%%%%%%%%%%%%%%%%%%%%%%
%Kommittéer
%%%%%%%%%%%%%%%%%%%%%%%%%%%%%%%%
\section{Kommittéer}
\label{sec:kommitteer}

\subsection{}
Ordförande i kommitté är skyldig att
\begin{itemize}
\item Kontinuerligt meddela kommitténs ekonomiska status till styret
\item Tillsammans med kommitténs kassör ansvara för att kommittén förvaltar sina tillgångar i enlighet med sektionens stadgar, reglementen och beslut.
\item Tillsammans med kommitténs kassör skriva ett bindande avtal med styret angående skuldfrågan vid felaktig bokföring
\end{itemize}
\subsection{}
\label{sec:kommittee_kassor}
Kassör i kommitté är skyldig att
\begin{itemize}
\item Föra kassabok av sådan typ som godkänts av sektionens revisorer
\item Tillsammans med kommitténs ordförande ansvara för att kommittén förvaltar sina tillgångar i enlighet med sektionens stadgar, reglementen och beslut.
\item Tillsammans med kommitténs ordförande skriva ett bindandeavtal med Styret angående skuldfrågan vid felaktigt förd bokföring
\item På varje sektionsmöte redovisa kommitténs ekonomiska situation
\item Arkivera kommitténs bokföring, på en plats anvisad av Styret, så lång tid som föreskrivs för den organisationsform som datateknologsektionen är.
\item Lägga budget enligt \S\ref{sec:budget}
\end{itemize}
\subsection{}
För kommittéer som saknar kassör ansvarar istället styrelsens kassör för samtliga av punkterna i \S\ref{sec:kommittee_kassor} med undantag för att lägga budget vilket görs i samråd med kommittéens ordförande
\subsection{Budget}
\label{sec:budget}
Varje kommitté ska i början av året lägga en budget för hur dess pengar ska spenderas. Det är upp till styrelsens kassör att delge en mall för budget för att underlätta kommittéernas arbete. I budgeten ska det tydligt framgå hur pengarna ska spenderas med en uppskattning för vilken månad utgifter och inkomster hamnar i.

\subsubsection{}
Varje kommittés budgetförslag ska godkännas av styrelsen. Som rapporterar vidare relevant information till sektionsmötet.


\subsection{Kapital}
\label{sec:sektionsforeningar_startkapital}
\subsubsection{Startkapital}
Kommittéer skall vid mandatperiodens början inte ha några tillgångar.
\subsubsection{Startlån}
Följande Kommittéer har rätt att låna \prisbasbelopp{0.25} från sektionsstyrelsen i form av startlån:
\begin{itemize}
  \item DRust
  \item DAG
  \item Delta
  \item D6
  \item DNollK
  \item DBus
  \item DKock
\end{itemize}
DAG har rätt låna ytterligare \prisbasbelopp{1} i form av utökat startlån.
\subsubsection{Slutkapital}
Samtliga kommitteers tillgångar och skulder tillfaller sektionsstyrelsen vid mandatperiodens slut.
\subsubsection{Stödkapital}
Styret har rätt att utföra kortfristiga lån till kommittéer vid behov. Dessa ska isåfall betalas tillbaka senast i slutet av mandatperioden.

%%%%%%%%%%%%%%%%%%%%%%%%%%%%%
% Äskning
%%%%%%%%%%%%%%%%%%%%%%%%%%%%%
\section{Äskning}
\subsection{Till styrelsen}
Medlem eller kommittéer som önskar extra ekonomiska medel till sin verksamhet eller arrangemang skall inkomma med önskemål och skäl till styret.
Styret kan bevilja extra medel om beloppet faller inom ramarna för de olika fonderna som styrelsen förvaltar. Om äskningen inte fallar inom dessa ramar kan den ändå beviljas förutsatt att beloppet understiger \prisbasbelopp{0.25} och inte överstiger sektionens budget.
\subsection{Till sektionsmötet}
Medlem eller kommittéer kan också inkomma med önskemål om extra ekonomiska medel till sin verksamhet eller arrangemang genom motion till sektionsmötet. Finansieringen av motionen ska göras med prioritet på medel från de fonder som styrelsen förvaltar.

%%%%%%%%%%%%%%%%%%%%%%%%%%%%%%
% Förmåner
%%%%%%%%%%%%%%%%%%%%%%%%%%%%%%
\section{Rättigheter}
För rättigheter får inte pengar spenderas på alkohol om inte annat anges.

\subsection{Internrepresentation}
\label{sec:internreps}
Varje kommitté har inte rätt att belasta sin ekonomi med mer än:
\begin{itemize}
    \item[-] \prisbasbelopp{0.004} för teambuildingaktiviteter per person.
    \item[-] \prisbasbelopp{0.0020} (avrundat till närmaste femkrona) för överlämning per person som deltar från avgående samt påstigande år.
    \item[-] \prisbasbelopp{0.005}  per post för aspning och aspplagg sammanlagt.
\end{itemize}

\subsubsection{}
Upp till en tredjedel av beloppet för teambuildingaktiviteter får användas till alkohol per person i kommittén.

\subsubsection{}
För att få använda hela det budgeterade beloppet för teambuilding måste 50\% av beloppet användas inom 8 månader från mandatperiodens början. Annars får inte mer än det som spenderats alternativt upp till 25\% användas.

\subsubsection{}
Varje kommitée har rätt att skicka in en förfrågan för att använda upp till
0.017 prisbasbelopp per person i kommitéen för aspning och aspplagg sammanlagt. För att
få lov att spendera denna summa på aspning ska en kommitté få godkännande av
styrelsen innan inköp. Detta genom att kommittén skickar in plan för aspningen samt
en kostnadskalkyl som styrelsen baserar sitt beslut på.


\subsection{Representationskläder}
Varje kommitté har rätt att köpa representationskläder till varje medlem 
för att synliggöra sig själva på sektionen och campus. Representationskläder 
kan ses som, overaller, arbetsbyxor, hoodies, t-shirts eller liknande.
Kostnaden för representationskläder får inte överstiga 0.01 prisbasbelopp 
per person i kommittén.

\subsubsection{}
Varje kommitté har rätt att skicka in en förfrågan för att använda upp
till 0.022 prisbasbelopp per person i kommitéen. För att få lov att köpa 
in representationskläder för denna summa ska en kommitté få godkännande 
av styrelsen innan inköp. Detta genom att kommittén skickar förslag på 
kläder och en uppskattad kostnad som styrelsen baserar sitt beslut på.

\subsubsection{}
En ansökan om representationskläder ska innehålla:
\begin{itemize}
  \item Det som man vill köpa in och vad som tidigare har köpts in.
  \item När och i vilka sammanhang representationskläder kommer användas.
  \item Varför gruppen vill ha representationskläder.
  \item En ungefärlig kostnadsberäkning.
\end{itemize}

\subsubsection{}
När styrelsen beslutar huruvida ett förslag godkänns ska följande tas hänsyn till:
\begin{itemize}
    \item Att priset inte är onödigt dyrt (jämfört med liknande kläder för andra kommittéer).
    \item Att kläderna har en anknytning till verksamheten.
    \item Att kommittén kommer använda kläderna tillräckligt mycket för att motivera inköp.
\end{itemize}

\subsection{Representation \& Möteskostnader}
Vid arrangemang för sektionen kan representation eller möteskostnader beviljas. Dessa belastar kommittéens ekonomi. Hur mycket som godkänns baseras på arrangemangets längd där beloppet avrundas till närmaste femkrona:
\begin{itemize}
    \item 1 till 4 timmar: \prisbasbelopp{0.00075} per person.
    \item 4 till 8 timmar: \prisbasbelopp{0.0015} per person.
    \item Mer än 8 timmar: \prisbasbelopp{0.00225} per person.
\end{itemize}
\subsubsection{}
Vid redovisning av representation eller möteskostnader ska deltagare samt tiden de deltagit finnas med för att belopp ska kunna godkännas.

\subsection{}
Ovanstående belopp får inte överskrida kommitténs ekonomiska resurser. \\
%%%%%%%%%%%%%%%%%%%%%%%%%%%%%%%
% Sponsring
%%%%%%%%%%%%%%%%%%%%%%%%%%%%%%%
\section{Sponsring}
\subsection{}
Om kommittéer vill söka sponsring ska detta göras i samråd med datas arbetsmarknadsgrupp, DAG.
\section{Fonder}
De fonder som beskrivs i denna del syftar till sparande av medel för framtida användande enligt respektive fonds syfte, antingen planerat eller spontant. Ett uttag ur en fond skall inte belasta Sektionsstyrelsens godkända budget, däremot skall inköp och investeringar vara föremål för redovisning i balans– respektive resultaträkning. Detta gäller om inget annat specificeras i fonden.

\subsection{Uttag ur fonder}
Sektionsmötet äger alltid rätt att göra uttag ur fonder i enhet med fondens syfte. Sektionsmötet äger rätt att vid 2/3 majoritet göra uttag ur fonder i strid mot fondens syfte, ett välmotiverat behov till uttaget skall protokollföras vid ett sådant beslut. Alla andra uttagsformer måste specificeras i det ekonomiska reglementet.

\subsection{Redovisning av uttag}
Alla uttag ur fonder ska redovisas på sektionsmöte i styrelsens verksamhetsrapport.

\subsection{Kapitalfonden}
\subsubsection{Syfte}
\label{sec:kapitalfond_syfte}
Syftet med kapitalfonden är att avlasta sektionens respektive de olika sektionskommittéernas ekonomi från stora investeringar. Pengarna skall användas till saker som har ett bestående värde, samt är till gagn för sektionens medlemmar direkt eller indirekt. Pengar skall ej användas till driftbidrag eller stöd för förgänglig verksamhet. Fonden skall ej användas till verksamhet som lokalfond är ämnad för.
\subsubsection{Förvaltning}
Fonden förvaltas av sektionens styrelse.
\subsubsection{Avsättning}
\begin{itemize}
\item En av styrelsen budgeterad summa som godkänts av sektionsmötet
\item All avkastning ifrån kapitalfonden under verksamhetsåret.
\end{itemize}
\subsubsection{Uttag}
\begin{itemize}
\item Sektionsstyrelsen har rätt att bevilja uttag ur fonden på belopp upp till totalt \prisbasbelopp{0.25} per tillfälle. Uttag av belopp överstigande \prisbasbelopp{0.25} skall godkännas av sektionsmöte innan medel utbetalas.
\item Sektionsstyrelsen äger inte rätt att ta ut mer än 50\% av fondens totala värde per tillfälle.
\end{itemize}

\subsection{Lokalfonden}
\subsubsection{Syfte}
\label{sec:lokalfond_syfte}
Syftet med lokalfonden är att säkra medel för underhåll och reparationer av sektionens lokaler, samt för inköp av inventarier. Pengarna skall användas till större reparationer och ommålningar av sektionslokalerna, samt möbler till trivselytor där teknologen i gemen har tillträde.
\subsubsection{Förvaltning}
Fonden förvaltas av sektionens styrelse.
\subsubsection{Avsättning}
\label{sec:lokalfond_avsattning}
\begin{itemize}
\item Minst Tio (10)\% av under verksamhetsåret influtna sektionsavgifter tillförs lokalfonden.
\item En av styrelsen budgeterad summa som godkänts av sektionsmötet
\item All avkastning ifrån lokalfonden under verksamhetsåret.
\end{itemize}
\subsubsection{Uttag}
\begin{itemize}
\item Sektionsstyrelsen disponerar sjuttiofem (75) \% av årets tillförda kapital,
enligt \S\ref{sec:lokalfond_avsattning}, för basdrift av sektionslokalerna.
\item Sektionsstyrelsen har rätt att besluta om ytterliggare uttag, dock skall detta redovisas inför nästkommande sektionsmöte. Vid detta sektionsmöte skall i sådana fall även genomförda eller planerade inköp, reparationer och underhåll redovisas.
\end{itemize}

\subsection{Bilfonden}
\subsubsection{Syfte}
\label{sec:bilfond_syfte}
Syftet med bilfonden är att bygga upp en buffert för bilinköp och minska belastningen av sektionens ekonomi vid omfattande skador.
\subsubsection{Förvaltning}
Fonden förvaltas av sektionens styrelse.
\subsubsection{Avsättning}
\begin{itemize}
\item En av styrelsens budgeterad summa som godkänts av sektionsmötet.
\item All avkastning ifrån bilfonden under verksamhetsåret.
\end{itemize}
\subsubsection{Uttag}
\begin{itemize}
    \item Sektionsstyrelsen äger rätt att bevilja uttag för omfattande reparation av sektionensbil.
    \item Sektionsstyrelsen har rätt att köpa in en ny bil efter att ha fått en proposition till sektionsmötet godkänd med en specifikation på kostnad och krav för bilen.
\end{itemize}

\subsection{Idéfonden}
\subsubsection{Syfte}
\label{sec:idefond_syfte}
Syftet med idéfonden är att möjliggöra för datateknologer som önskar medel till arrangemang eller inventarier vars effekt är till gagn för hela teknologsektionens medlemmar.
\subsubsection{Förvaltning}
Idéfonden förvaltas av sektionsstyrelsen.
\subsubsection{Avsättning}
\begin{itemize}
\item Tio (10) \% av under verksamhetsåret influtna sektionsavgifter tillförs idéfonden.
\item En av styrelsen budgeterad summa som godkänts av sektionsmötet.
\item All avkastning ifrån idéfonden under verksamhetsåret.
\end{itemize}
\subsubsection{Uttag}
\begin{itemize}
  \item Sektionsstyrelsen har rätt att, på förslag av medlem i teknologsektionen, bevilja uttag på upp till 0,25 prisbasbelopp som går i hand med idéfondens syfte.
  \item Sektionsmötet har rätt att, på förslag av medlem i teknologsektionen, bevilja uttagsom går i hand med \S\ref{sec:idefond_syfte}.
\end{itemize}

\subsection{Husfonden}
\subsubsection{Syfte}
\label{sec:husfond_syfte}
Syftet med husfonden är att möjliggöra framtida fastighetsinvestering eller lokalför-värvning för sektionensmedlemmars bästa.
\subsubsection{Förvaltning}
Husfonden förvaltas av sektionsstyrelsen.
\subsubsection{Avsättning}
\begin{itemize}
  \item En av styrelsen budgeterad summa som godkänts av sektionsmötet.
  \item All avkastning ifrån husfonden under verksamhetsåret.
\end{itemize}
\subsubsection{Uttag}
\begin{itemize}
  \item Sektionsstyrelsen har rätt att göra uttag som uppfyller syftet efter att ha fått en proposition till sektionsmötet godkänd som inkluderar en kostnadskalkyl och annan relevant information om investeringen.
\end{itemize}
\subsection{Jubileumsfonden}
\subsubsection{Syfte}
\label{sec:jubileumfond_syfte}
Syftet med Jubileumsfonden är att möjliggöra firande av sektionens framtida jubileum, genom att möjliggöra större och dyrare evenemang.
\subsubsection{Förvaltning}
Jubileumsfonden förvaltas av sektionsstyrelsen.
\subsubsection{Avsättning}
\begin{itemize}
  \item En av styrelsen budgeterad summa som godkänts av sektionsmötet.
  \item All avkastning ifrån Jubileumsfonden under verksamhetsåret.
\end{itemize}
\subsubsection{Uttag}
\begin{itemize}
  \item Sektionsstyrelsen har rätt att göra uttag under jubileumsår.
\end{itemize}
\end{document}
