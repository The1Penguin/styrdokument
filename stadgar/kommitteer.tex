\section{Kommittéer}
\subsection{Definition}
\subsubsection{Ledamöter}
Kommitté ska ha ordförande och ett i reglementet fastställt antal förtroendeposter.

\subsubsection{Uppgift}
Kommitté ska verka för teknologsektionens bästa och ha en i reglementet fastställd uppgift.

\subsection{Val}
\subsubsection{Invalsprocess}
Samtliga förtroendeposter tillsätts av sektionsmötet om inte annat bestäms i reglementet.
\subsubsection{Valberedningsprocess}
Val till kommitté kan beredas av valberedning enligt~\ref{sec:valberedning}.

\subsubsection{Valbarhet}
Valbar till ledamot i kommitté gäller de som nämns i~\ref{sec:medlemmar_valbarhet}.
Valbarhet till ordförande och kassör kräver myndighet.
Ytterligare bestämmelser för valbarhet kan finnas i reglementet.

\subsection{Rättigheter}
Kommitté äger rätt att i namn och emblem använda teknologsektionens namn och dess symboler i enlighet med Chalmers Studentkårs policyer.
\subsection{Skyldigheter}
Kommitté är skyldig att rätta sig efter teknologsektionens stadga, reglemente, ekonomiska reglemente, policyer och övriga fattade beslut.
\subsection{Revision}
Kommittéernas verksamhet och ekonomi granskas av teknologsektionens lekmannarevisorer.
\subsection{Förteckning}
Teknologsektionens kommittéer förtecknas i reglementet.
