\section{Intresseföreningar}
\subsection{Definition}
Intresseförening är en sammanslutning av sektionens medlemmar och stödmedlemmar med ett gemensamt intresse.
Intresseföreningen ska ha en styrelse bestående av föreningens medlemmar.
Ordförande i föreningsstyrelsen väljs av sektionsmötet.
\subsubsection{Medlemsrätt}
Varje sektionsmedlem skall ha rätt till medlemskap.
Föreningsmedlem som motverkar föreningens syften kan dock uteslutas av föreningsstyrelsen.
\subsubsection{Stadga}
Intresseförening skall ha en av sektionsstyrelsen godkänd stadga.
Dessutom är föreningen skyldig att rapportera till sektionsstyelsen då deras stadga förändrats.
\subsubsection{Syfte}
Intresseförening skall verka för teknologsektionens bästa och ha ett syfte i sin egen stadga.
\subsection{Förteckning}
Teknologsektionens intresseföreningar är de i reglemente förtecknade.
\subsection{Rättigheter}
Intresseförening äger rätt att i namn och emblem använda teknologsektionens namn och symboler.
\subsection{Skyldigheter}
Intresseförening är skyldig att känna till och rätta sig efter teknologsektionens stadgar, reglemente, ekonomiska reglemente, policies, övriga handlingar och beslut.
\subsection{Ekonomi}
Intresseföreningens ekonomi skall ligga under teknologsektionen.
\subsubsection{Verksamhet och Revision}
Teknologsektionens revisorer har rätt att granska föreningens verksamhet och ekonomi.
