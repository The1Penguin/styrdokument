\section{Avsättning}
\subsection{Behöriga}
Sektionsmötet kan avsätta individer invalda av sektionsmötet, sektionsstyrelsen eller intresseförenings medlemmar.
Avsättning kan endast ske efter att behörig begäran om avsättning hanteras av sektionsmötet.
\subsection{Begäran}
Begäran om avsättning sker genom styrelsebeslut med en kvalificerad majoritet av styrelsens ordinarie ledamöter, skriftlig begäran från 25 medlemmar eller av inspektor.
Begäran om avsättning av sektionsstyrelsen eller sektionsstyrelseledamot ska skickas till Talhenspresidiet.
I övriga fall ska begäran skickas till sektionsstyrelsen och Talhenspresidiet.
\subsection{Yttrande}
Personen som begäran avser ska ges möjlighet att yttra sig i frågan under sektionsmötet.
\subsection{Votering}
För att bifalla begäran om avsättning krävs det en strikt kvalificerad majoritet.
Omröstningen ska alltid ske med sluten votering.
\subsection{Interimsstyrelese}
Vid sektionsstyrelsens avsättande ska interimsstyrelse och ny valberedning väljas.
Interimssektionsstyrelsen utfärdar kallelse till extra sektionsmöte där ny ordinarie sektionsstyrelse ska väljas för resten av mandatperioden.
Detta sektionsmöte ska hållas inom 15 läsdagar från det sektionsmöte då interimssektionsstyrelsen valdes och under ordinarie terminstid.

Interimsstyrelsen övertar sektionsstyrelsens befogenheter och skyldigheter tills dess en ny sektionsstyrelse är vald.
\subsection{Interims-DNS}
Vid hela DNS avsättande ska interims-DNS väljas.
Vid näst kommande sektionsmöte väljs ett nytt ordinarie DNS för resten av mandatperioden.

Interims-DNS övertar DNS befogenheter och skyldigheter tills dess ett nytt DNS är valt.
\subsection{Interims-lekmannarevisorer}
Vid avsättande av samtliga lekmannarevisorer ska Interims-lekmannarevisorer väljas.
Vid näst kommande sektionsmöte väljs nya ordinarie lekmannarevisorer för resten av mandatperioden.

Interims-lekmannarevisorer övertar lekmannarevisorernas befogenheter och skyldigheter tills dess att nya lekmannarevisorer är valda.

