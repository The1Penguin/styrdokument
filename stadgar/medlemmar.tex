\section{Medlemmar}
\label{sec:medlemmar}
Medlem i teknologsektionen är den som är inskriven vid utbildningsprogrammet Datateknik civilingenjör vid Chalmers eller ett av dess associerade masterprogram. Dessutom ska medlem ha erlagt sektionsavgift. Därutöver kan teknologsektionen ha hedersmedlemmar och stödmedlemmar.
\subsection{Rättigheter}
\subsubsection{Sektionsmöte}
Varje medlem har närvaro-, yttrande-, förslags-, och rösträtt på sektionsmöte.
\subsubsection{Valbarhet}
\label{sec:medlemmar_valbarhet}
Endast medlem är valbar till förtroendepost inom teknologsektionen. Lekmannarevisorerna och inspektor är undantagna föregående regel. Förtroendepost innebär vald av sektionsmötet eller sektionsstyrelsen.
\subsubsection{Handlingar}
Medlem har rätt att ta del av mötesprotokoll och teknologsektionens övriga handlingar.
\subsection{Skyldigheter}
Medlem är skyldig att rätta sig efter teknologsektionens bestämmelser.
%Hedersmedlemmar%%%%%%%%%%%%%%%%%%%%%%%%%%%%%%%%%%%%%%%%%%%%%%%%%%%%%%%
\subsection{Hedersmedlemmar}
\subsubsection{Grundkrav}
För att kunna bli hedersmedlem bör personen synnerligen främjat sektionens intressen och strävande.
\subsubsection{Antagande}
Om personen blivit föreslagen till sektionsstyrelsen och Talhenspresidiet 7 läsdagar före näst kommande sektionsmötet beslutas det om genom en kvalificerad majoritet.
Förslaget ska inkludera 25 underteckningar av sektionsmedlemmar.
\subsubsection{Rättigheter}
Hedersmedlem har närvaro- och yttranderätt på sektionsmöte.

\subsection{Stödmedlemmar}
\subsubsection{Grundkrav}
För att kunna bli stödmedlem måste personen tidigare varit medlem på sektionen.
\subsubsection{Antagande}
Om personen erlagt en administrationsavgift till sektionen blir personen en Stödmedlem.
\subsubsection{Rättigheter}
Stödmedlemmar har närvarorätt på sektionsmöte. Stödmedlem har dessutom rätt att ta del av mötesprotokoll och teknologsektionens övriga handlingar.
