\section{Revision, Inspektion och ansvarsfrihet}
\subsection{Lekmannarevisorer}
\subsubsection{Val \& Uppgift}
Sektionsmötet utser 2–4 lekmannarevisorer med uppgift att granska teknologsektionens verksamhet och ekonomi under verksamhetsåret.
\subsubsection{Kriterier}
Teknologsektionens lekmannarevisorer kan ej inneha annan förtroendepost på teknologsektionen under sitt verksamhetsår.
\subsubsection{Revision}
Räkenskaper och övriga handlingar ska tillställas lekammanrevisorerna senast 15 läsdagar före sektionsmöte.
\subsection{Åligganden}
\subsubsection{Åligganden}
Det åligger lekmannarevisorerna att skicka in revisionsberättelser till talhenspresidiet senast 5 läsdagar före ordinarie sektionsmöte.
\subsubsection{Berättelse}
Revisionsberättelsen ska innehålla yttrande ifråga om ansvarsfrihet för berörda personer.
\subsection{Ansvarsfrihet}
\subsubsection{Beslut}
Ansvarsfrihet är beviljad berörda personer då sektionsmötet fattat beslut om detta.
\subsubsection{Undantag}
Skulle förtroendevald på teknologsektionen med ekonomiskt ansvar avgå före mandatperiodens slut, ska revision företagas.
\subsection{Inspektor}
\subsubsection{Allmänt}
Inspektor ska ägna uppmärksamhet åt och stödja teknologsektionens verksamhet. Inspektor ska därvid hållas underrättad om teknologsektionens verksamhet. Inspektor har rätt att ta del av teknologsektionens protokoll och övriga handlingar.
\subsubsection{Val}
Inspektor väljs av sektionsmötet för en tid av två kalenderår.
